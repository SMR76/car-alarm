\opchapter{پیوست‌ها}

این فصل شامل بخشی از کد‌های نوشته شده به همراه توضیحاتی از آن‌ها است.
مطابق توضیحات در فصل‌های قبلی، پروژه شامل سه بخش بوده که در مجموع از ۶ تا ۷ زبان برنامه‌نویسی مجزا تشکیل شده.
سعی بر اینست که هر بخش به صورت شفاف توضیح داده شود.
\section{نرم‌افزار}
\subsection{درخواست \lr{Post} کلاس \lr{C++}}
فایل اصلی:
\hyperref{https://github.com/cardianco/cardian/blob/master/cardian/requesthandler.h}{}{}{\lrm{cardian/requesthandler.h}}

مورد استفاده: ‌ارسال درخواست‌های \lr{post}.
\begin{latin}
	\small
	\begin{lstlisting}[language=C++, caption={C++ Request Handler Component}]
		/// Implementation details of the main code.
		class requestHandler : public QObject {
			Q_OBJECT
			QML_NAMED_ELEMENT(RequestHandler)
			QML_ADDED_IN_MINOR_VERSION(1)
		public:
			Q_INVOKABLE int postRequest(const QUrl &url, const QByteArray &body, const QVariantMap &extraHeads = QVariantMap(), int timeout = DefaultTransferTimeoutConstant) {
				QNetworkRequest req(url);
				req.setTransferTimeout(timeout);
				req.setHeader(QNetworkRequest::ContentTypeHeader, "application/json");

				for(const auto &head: core::QKVRange(extraHeads)) {
					req.setRawHeader(head.first.toLatin1(), extraHeads[head.first].toByteArray());
				}

				QNetworkReply *reply = mNetworkManager.post(req, body);
				mRequestHash[reply].id = ++mNum;

				emit runningChanged();

				QObject::connect(reply, &QNetworkReply::readyRead, this, &requestHandler::onReadyRead);
				QObject::connect(reply, &QNetworkReply::finished, this, &requestHandler::onFinished);
				QObject::connect(reply, &QNetworkReply::errorOccurred, this, &requestHandler::onErrorOccurred);

				return mNum;
			}

		private:
			QNetworkAccessManager mNetworkManager;
			QHash<QNetworkReply *, replyData> mRequestHash; // The container used for holding each request
			int mNum; /// This is only for counting requests.
		};
	\end{lstlisting}
\end{latin}

\subsection{کلاس \lr{secureKeyChain}}
مورد استفاده: ذخیرهٔ ایمن \lr{token}های کاربر توسط در \lr{keychain}.

آدرس فایل:
\hyperref{https://github.com/cardianco/cardian/blob/master/cardian/securekeychain.h}{}{}{\lrm{cardian/securekeychain.h}}
\begin{latin}
	\small
	\begin{lstlisting}[language=C++, caption={C++ KeyChain Component}]
		/// Implementation details of the main code.
		class secureKeyChain: public QObject, public QQmlParserStatus {
			Q_OBJECT
			Q_PROPERTY(QVariantMap cache READ cache NOTIFY cacheChanged FINAL)

			QML_NAMED_ELEMENT(SecureKeyChain)
			QML_ADDED_IN_MINOR_VERSION(1)
		public:
   			QVariant readKey(const QString &key) {
				if(mCache.contains(key)) return mCache[key];

				credentialReader *reader = new credentialReader(service, this);
				connect(reader, &credentialReader::finished, this, &secureKeyChain::readerFinished);

				reader->setKey(key);
				reader->setAutoDelete(true);
				reader->start();

				return QVariant{}; /// Return undefined
			}
		private:
			QVariantMap mCache;
		};
	\end{lstlisting}
\end{latin}

\subsection{کلاس \lr{ٰVeqtor}}
مورد استفاده: کلاس پایهٔ استفاده شده برای نمایش و رسم شکل‌ها و چارت‌ها.

آدرس فایل:
\hyperref{https://github.com/0smr/veqtor/veqtor.h}{}{}{\lrm{cardian/third-parties/veqtor/veqtor.h}}
\begin{latin}
	\small
	\begin{lstlisting}[language=C++, caption={C++ ٰVeqtor Root Class}]
		/// Implementation details of the main code.
		class veqtor : public QNanoQuickItem {
			Q_OBJECT
			QML_NAMED_ELEMENT(Veqtor)
			QML_ADDED_IN_MINOR_VERSION(1)

			Q_PROPERTY(QString src READ src WRITE setSrc NOTIFY srcChanged)
			Q_PROPERTY(QVariantMap document READ document NOTIFY documentChanged)
			Q_PROPERTY(QObject* root READ root NOTIFY rootChanged)
			Q_PROPERTY(QSizeF sourceSize READ sourceSize CONSTANT)
		public:
			veqtor(QQuickItem *parent = nullptr);
			QNanoQuickItemPainter *createItemPainter() const override final;

			void hoverMoveEvent(QHoverEvent* event) override;

			void componentComplete() override;
			void painter(QNanoPainter *painter) const;

			QPointer<elements::svg> root() { return mRoot; }

			QString src() const { return mSrc; }
			void setSrc(const QString& src);

			QVariantMap document() const { return mDocument; }

			Q_INVOKABLE QVariant getElementById(const QString &id) { return mDocument[id]; }

			QSizeF sourceSize() const { return mSourceSize; }

		private slots:
			void setElementsToProperties();
			void adjustSize();
			void adjustResponsive();
			void updateDocument(const QString &newId, const QString &oldId);
			void updateElementAttributes();

		public slots:
			void update();

		signals:
			void srcChanged();
			void rootChanged();
			void documentChanged();
			void svgLoaded();
			void hovered(QPointer<elements::element> target);

		private:
			QPointer<elements::svg> mRoot;
			QVariantMap mDocument;
			QString mSrc;
			QSizeF mSourceSize;

			QTimer mUpdateTimer;
			QTransform mAdjustment;
		};
	\end{lstlisting}
\end{latin}

\subsection{فایل \lr{Statuses}}
مورد استفاده: نگهداری، و دریافت داده‌ها از سرور.

آدرس فایل:
\hyperref{https://github.com/cardianco/cardian/blob/master/cardian/globals/Status.qml}{}{}{\lrm{cardian/globals/Status.qml}}
\begin{latin}
	\small
	\begin{lstlisting}[language=qml, caption={Global Status Component}]
		/// Implementation details of the main code.
  		RequestHandler {
			id: requestHandler
			property var responses: Object();

			function graphAPI(body) {
				// Config.api == https://cardianWebsite/graphQL.php
				// Config.token == <64 byte hex string> == SHA256
				return postRequest(Config.api, JSON.stringify(body), {stoken: Config.token})
			}

			onErrorOccurred: /* Handling errors */;
			onFinished: function(data, id) {
				if(typeof responses[id].callback == 'function') {
					const args = responses[id].args ?? [];
					responses[id].callback(data, ...args);
					delete responses[id];
				}
			}
		}

		function fetchStatuses() {
			const limit = 100, start = events.first() ?? {};
			const body = { query: `query{statusHistory(limit:${limit},start:${start.id ?? -1}){id fieldType{id name} data utc}}` };
			const id = requestHandler.graphAPI(body);

			reqHndlr.responses[id] = {
				callback: result => {
					const history = JsUtils.parseJson(result, {}).qget('data.statusHistory', []);
					if(history) {
						history.forEach(record => {
							const {id, fieldType, data, utc} = record;
							/// Here, `events` refers to an object instance of `EventModel`.
							events.append(id, fieldType.name, data, utc, fieldType.id);
						});
					}
				}
			};
		}
	\end{lstlisting}
\end{latin}


\subsection{صفحهٔ پیمایش}
مورد استفاده: بخشی از کد صفحهٔ‌ پیمایش نرم‌افزار.

آدرس فایل:
\hyperref{https://github.com/cardianco/cardian/blob/master/cardian/pages/NavigationPage.qml}{}{}{\lrm{cardian/pages/NavigationPage.qml}}
\begin{latin}
	\small
	\begin{lstlisting}[language=qml, caption={Navigation Page}]
		/// Implementation details of the main code.
		RouteModel {
			id: routeModel

			autoUpdate: true
			query: routeQuery

			plugin: Plugin {
				name: "mapbox"
				/// `control.token` here refered as mapboxGL access token.
				PluginParameter { name: "mapbox.access_token"; value: control.token }
			}
		}

		RouteQuery { id: routeQuery }

		contentItem: Map {
			id: map

			opacity: 1.0
			zoomLevel: 14
			copyrightsVisible: false
			center: QtPositioning.coordinate(36.3162, 59.5408) // Mashhad (Default location, clould also be Tehran)
			color: "transparent"
			gesture.enabled: !dragPoly.dragging

			plugin: Plugin {
				name: 'mapboxgl'
				locales: ['en_US' ,'fa_IR']
				preferred: ["mapboxgl", 'osm', "mapbox"]

				PluginParameter {
					name: 'mapboxgl.access_token'
					value: control.token
				}

				PluginParameter {
					name: 'mapboxgl.mapping.cache.memory'
					value: false
				}

				PluginParameter {
					name: 'mapboxgl.mapping.additional_style_urls'
					value: 'qrc:/resources/mapgl/dark-style.json'
				}
			}

			TapHandler {
				id: tapInserter
				grabPermissions: TapHandler.ApprovesTakeOverByAnything
				onTapped: event => {
					const {longitude, latitude} = map.toCoordinate(event.position);
					const {i} = JsUtils.findNearestPoint(Qt.vector2d(longitude, latitude), activePolygon);
					activePolygon.insert(i, {longitude, latitude});
				}
			}

			MapItemView {
				model: routeModel

				delegate: MapRoute {
					route: model.routeData
					line.color: "#aaF5478D"
					line.width: map.zoomLevel > 5 ? map.zoomLevel/2.8 : 0
					smooth: true
					opacity: index ? 0.3 : 1.0
				}
			}

			MapItemView {
				model: polygons

				delegate: MapItemGroup {
					required property int index
					required property var poly

					MapPolygon {
						opacity: 0.3
						color: index === Config.selectedMap ? control.palette.text : control.palette.highlight
						border.color: control.palette.window
						border.width: 1

						path: Array(poly.count).fill().map((_, i) => poly.get(i));
					}

					MapQuickItem {
						anchorPoint.x: sourceItem.width/2
						anchorPoint.y: sourceItem.height/2
						coordinate: {
							const {y, x} = JsUtils.polygonCenter(poly);
							return QtPositioning.coordinate(y, x);
						}
						sourceItem: Label {
							text: index + 1; font: Fonts.head
							color: control.palette.buttonText
							opacity: 0.6
						}
					}
				}
			}

			/// Dragable polygons on navigation page
			MapDragPolygon { /* Commented out */ }

			MapQuickItem {
				coordinate: control.currentUserLocation ?? QtPositioning.coordinate()
				anchorPoint{x: 7.5; y: 7.5}
				sourceItem: Rectangle { /* Commented out */ }
			}

			MapQuickItem {
				coordinate: control.currentCarLocation ?? QtPositioning.coordinate()
				anchorPoint{x: sourceItem.width / 2; y: sourceItem.height / 2}
				sourceItem: SomeIcon { /* Commented out */ }
			}

			Shortcut { /// Shortcuts key for desktop outputs.
				enabled: map.zoomLevel < map.maximumZoomLevel
				sequence: StandardKey.ZoomIn
				onActivated: map.zoomLevel = Math.round(map.zoomLevel + 1);
			}

			Shortcut {
				enabled: map.zoomLevel > map.minimumZoomLevel
				sequence: StandardKey.ZoomOut
				onActivated: map.zoomLevel = Math.round(map.zoomLevel - 1);
			}
		}
	\end{lstlisting}
\end{latin}

% ------------------
\subsection{.}
مورد استفاده: بخش تنظیمات پیشرفته.
شامل نمایش و کنترل دقیق‌تر اجزاء خودرو.

آدرس فایل:
\hyperref{https://github.com/cardianco/cardian/blob/master/cardian/pages/ExtraPage.qml}{}{}{\lrm{cardian/pages/ExtraPage.qml}}
\begin{latin}
	\small
	\begin{lstlisting}[language=qml, caption={ExtraPage.qml}]
		/// Implementation details of the main code.
		BasePage {
			id: page
			title: swipeview.currentItem.title

			contentItem: Column {
				padding: 5
				topPadding: 35

				TabBar {
					id: tabbar
					x: (parent.width - width)/2
					currentIndex: swipeview.currentIndex

					Repeater {
						model: ["Advanced", "Configuration", "Events"]
						TabButton { /* Commented out */ }
					}
				}

				SwipeView {
					id: swipeview

					width: page.width - 2 * parent.padding
					height: page.height - y - 10

					topPadding: 10
					clip: true

					spacing: parent.padding * 2
					currentIndex: tabbar.currentIndex

					Advanced {
						oriention: page.oriention
						title: "Advanced Controls"
					}

					Configuration {
						oriention: page.oriention
						title: "Configuration"
					}

					Events {
						oriention: page.oriention
						title: "Commands Events"
					}
				}
			}
		}
	\end{lstlisting}
\end{latin}

\section{سرور و \lr{API}}

\subsection{\lr{Schema}}
تعریف طرح دادهٔ \lr{GraphQl}
در این بخش نوع داده‌های مورد استفاده در پرس‌وجوی \lr{GraphQL} استفاده شده‌اند تعریف شده است.

آدرس فایل:
\hyperref{https://github.com/cardianco/cardian-server/blob/master/src/api/graphql/schema.graphql}{}{}{\lrm{cardian-server/src/api/graphql/schema.graphql}}
\begin{latin}
	\small
	\begin{lstlisting}[language=graphql, caption={GraphQL Schema.}]
		/// Implementation details of the graphQL schema.
		type Mutation {
			newStatus(fieldId: ID!, value: String!): ID
			...
		}

		type Query {
			fields: [Field]
			states: [State]
			deviceTypes: [DeviceType]

			user: User
			session: Session
			sessions: [Session!]
			latestStatus: LatestStatus!
			statusHistory(limit: Int! = 1, start: Int = -1, accessed: Boolean = null): [Status!]
			commandHistory(limit: Int! = 1, start: Int = -1, accessed: Boolean = null): [Command!]

			/// An example definition of type is provided below.
			boundaries(start: Int! = -1): [Boundary!]
		}

		/// Input type
		input PointInput {
			x: Float!
			y: Float!
		}

		type Point {
			x: Float!
			y: Float!
		}

		type Boundary {
			id: ID!
			session: Session!
			state: State!
			poly: [Point!] // The `Point` type was defined before.
			utc: Int!
		}
		/// ...
	\end{lstlisting}
\end{latin}

\subsection{\lr{Resolver}}
نمونهٔ خلاصه شده از کلاس \lr{resolver}، این کلاس یک پیاده‌سازی برای استفاده در کتاب‌خانهٔ \lr{webonyx/grapql} است.
که نحوهٔ عملکرد برای هر درخواست در آن مشخص شده‌است.

آدرس فایل:
\hyperref{https://github.com/cardianco/cardian-server/blob/master/src/api/graphql/resolver.graphql}{}{}{\lrm{cardian-server/src/api/graphql/resolver.graphql}}
\begin{latin}
	\small
	\begin{lstlisting}[language=php, caption={Rsolver Class}]
		/// Implementation details of the main code.
		class resolver {
			static public function values(database $db) {
				$typeResolvers = [
					'Session' => /* Implementations are commented out */ ,
					'LatestStatus' => /* Implementations are commented out */,
					'User' => function($values, $args, $ctx, $inf) use ($db) { /* Commented out */ },
					'Field' => fn($values, $args, $ctx, $inf) =>  $db->getFields($values['fid'] ?? null),
					'State' => fn($values, $args, $ctx, $inf) =>  $db->getStates($values['stid'] ?? null),
					'DeviceType' => fn($values, $args, $ctx, $inf) =>  $db->getDeviceType($values['tid'] ?? null),
					'UserInfo' => fn($values, $args, $ctx, $inf) =>  $db->getUserInfo($values['id'])
				];

				$rootResolvers = [
					'fields' => $typeResolvers['Field'],
					'states' => $typeResolvers['State'],
					'deviceTypes' => $typeResolvers['DeviceType'],

					'user' => $typeResolvers['User'],
					'session' => $typeResolvers['Session'],
					'latestStatus' => $typeResolvers['LatestStatus'],
					'sessions' => function($root, $args, $ctx, $inf) use ($db) { /* Commented out */  },
					'statusHistory' => function($root, $args, $ctx, $inf) use ($db) { /* Commented out */  },
					'commandHistory' => function($root, $args, $ctx, $inf) use ($db) { /* Commented out */  },
					'boundaries' => function($root, $args, $ctx, $inf) use ($db) {
						$start = $args['start']; /// Start index
						$bndry = $db->getBoundariesByUser($root['uid'], $start);

						foreach($bndry as &$b) $b['session'] = $root['Session'];
						foreach($bndry as &$b) $b['state'] = $root['State'];

						return $bndry;
					}
				];

				$mutations = [
					'newStatus' => function($root, $args, $ctx, $inf) use ($db) {
						[$fid, $value] = [$args['fieldId'], $args['value']];
						$field = $db->getFieldsPair()[$fid] ?? "json";

						$jsonString = json_encode([$field => json_decode($value)]);

						$db->updateLatestStatus($root['uid'], $jsonString);
						return $db->addStatus($root['sid'], $fid, $jsonString);
					},
					/* Other mutations are Commented out */
				];

				return array_merge($typeResolvers, $rootResolvers, $mutations);
			}
		}
	\end{lstlisting}
\end{latin}

\section{سخت‌افزار}
بخش سخت‌افزار شامل کتابخانه‌هایی می‌شوند که برای هرچه آسان‌تر بکارگیری ماژول‌های سخت‌افزاری طراحی شده‌اند.
که شامل کتابخانه‌های هستهٔ‌ برنامه مانند: \lr{attachable} و \lr{timer} هستند.
\begin{itemize}[nosep]
	\item \lr{attachable}،
	این کلاس در عمل نقش یک \lr{event loop} را دارد.
	بدین صورت که با تابع
	\begin{latin}
		\small
		\begin{lstlisting}[language=c++]
			on(const std::string \&, const std::function<void(const std::vector<std::any>\&));
		\end{lstlisting}
	\end{latin}
	می‌توان یک رخداد ایجاد کرد. \newline
	برای مثال برای
	\lrm{on("event-name", [](auto) { /* works to-do */ })}
	در صورت، فراخوانی تابع
	\lrm{emit("event-name");}
	تابع تعریف متصل شده توسط تابع
	\lrm{on}
	فراخوانی می‌شوند.
	\item \lr{timer}
	کلاس \lr{timer}،‌ که با وجود تابع
	\begin{latin}
		\small
		\begin{lstlisting}[language=c++]
			void timer::every(const std::function<void(void));
		\end{lstlisting}
	\end{latin}
	می‌توان یک تابع را در زمان مشخص شده فراخوانی کرد.
\end{itemize}

\begin{figure}[!h]
	\begin{center}
		\includegraphics[width=0.8\textwidth]{images/timer-instructure.pdf}
	\end{center}
	\caption{نمایش ساختار عملکرد \lr{timer}.}
	\label{fig1:chap6}
\end{figure}

\subsection{کلاس \lr{attachable}}
کلاس \lr{attachable} برای استفاده عنوان کلاس پایه، ارث‌بری شده در توسط کلاس‌های
\lr{serial}،
\lr{sim800}،
\lr{neo6m}،
\lr{hmtrp}،
\lr{serial}.

آدرس فایل:
\hyperref{https://github.com/cardianco/cardian-hardware/blob/master/main/utils/attachable.h}{}{}{\lrm{cardian-hardware/hardware/main/utils/attachable.h}}
\begin{latin}
	\small
	\begin{lstlisting}[language=c++, caption={Attachable class implementation}]
		/// Implementation details of the main code.
		using _any_list = std::vector<_any>;
		using _callback = std::function<void(const _any_list&)>;

		struct event {
			enum Type {Default, Once, Exclusive};
			Type type = Type::Default;
			_callback func;
		};

		class attachable {
			public:
			void emit(const fmt &name, const _any_list &args = {}) {
				if(mEvents.count(name)) {
					std::vector<event> &elist = mEvents[name];

					for(event &e: elist) {
						std::invoke(e.func, args);
						if(e.type == event::Exclusive) break;
					}

					if(elist.size() && elist.front().type == event::Exclusive) {
						elist.erase(elist.begin());
					} else {
						auto last = std::remove_if(elist.begin(), elist.end(), [](event &e){
							return e.type == event::Once;
						});
						elist.resize(std::distance(elist.begin(), last));
					}
				}
			}

			void on(const fmt &name, const _callback &func, event::Type type = event::Default) {
				std::vector<event> &elist = mEvents[name];
				if(type == event::Exclusive) {
					auto iter = std::find_if(elist.begin(), elist.end(), [](event &e){ return e.type != event::Exclusive; });
					elist.insert(iter, { type, func });
				} else {
					elist.push_back({ type, func });
				}
			}

			void clear(const fmt &name) {
				mEvents[name].clear();
			}

			attachable& then(const _callback &func) {
				on("then", func, event::Exclusive);
				return *this;
			}

			attachable& fail(const _callback &func) {
				on("fail", func, event::Exclusive);
				return *this;
			}

			size_t count() const { return mEvents.size(); }

			static void _NONE(const _any_list&) {}

			private:
			std::map<fmt, std::vector<event>> mEvents;
		};

		inline static attachable _attachable;
	\end{lstlisting}
\end{latin}

\subsection{کلاس \lr{sim800l}}
کلاس \lr{sim800} برای استفاده از ماژول
\lr{SIM800L}
یک یک کلاس ارث‌بری شده از \lr{attachable} است و قابلیت اجرای دستورات را برپایهٔ رخداد و غیر همزمان ارائه می‌کند.

آدرس فایل:
\hyperref{https://github.com/cardianco/cardian-hardware/blob/master/main/utils/sim800.h}{}{}{\lrm{cardian-hardware/hardware/main/utils/sim800.h}}
\begin{latin}
	\small
	\begin{lstlisting}[language=c++, caption={}]
		/// Implementation details of the main code.
		class sim800l final: protected core::attachable {
		public:
			sim800l() {
				mSerial1.setTimeout(1000);
				mcu::timer::repeat(10, std::bind(&core::serial1::eventLoop, &mSerial1));

				mSerial1.onReadyRead([this](auto) {
					response res{};

					fmt value = mSerial1.read();

					value = value.tolower();

					/// Reading all lines...

					if(value.includes("ok")) { res.type = response::OK; }
					else if(value.includes("error")) { res.type = response::ERROR; } else if(value.includes("call ready") || value.includes("sms ready")) { res.type = response::STATUS; }
					else { res.type = response::OK; }
					res.data = value;

					if(mResponses.isFull()) mResponses.remove(res);
					mResponses.insert(res);

					emit("response", {res});
				});
			}

			inline void onResponse(const core::_callback &cb, core::event::Type type = core::event::Once) {
				on("response", cb, type);
			}

			core::attachable& run(fmt command = "at") {
				mSerial1.print(command + "\r\n");

				static core::attachable _a;

				onResponse([](const core::_any_list &args){
					auto res = args[0].cast<response>().value();

					if(res.type == response::OK) {
						_a.emit("then", {res.data});
					} else {
						_a.emit("fail", {res.data});
					}
				});

				return _a;
			}

			void call(std::string pn) { run(fmt("ATD{0};").arg(pn)); }
			void endCall() { run("ATH"); }
			void sms(const std::string &phoneNumber, const std::string &message) {
				run("AT+CMGF=1")
				.then([this, &phoneNumber, &message](auto) {
					run(fmt(R"(AT+CMGS="{0}"\n{1}\x1a)").arg(phoneNumber).arg(message));
				});
			}

			void reset() {
				run("AT+CFUN=0").then([this](auto) {
					mcu::timer::once(30000, [this] {
						/// o: OK
						run("AT+CFUN=1");
					});
				});
			}

			void getCCID() { run("AT+CCID"); }
			void checkPin() { run("AT+CPIN?"); }
			void checkRegistration() { run("AT+CREG?"); }
			void checkGprsConnection() { run("AT+CGATT?");
			void detachGprs() { run("AT+CGATT=0"); }
			void checkHttpStatus() { run("AT+HTTPSTATUS?"); }
			void checkPhoneState() { run("AT+CPAS"); }
			void checkSignal() { run("AT+CSQ"); }
			void checkRoaming() { run("AT+CROAMING"); }

			void enableTcp() {
				run(R"(AT+CSTT="mtnirancell","","")").then([this](auto) {
					run("AT+CIICR");
				});
			}

		    void sendTcpData(const std::string &base, const std::string &url, const std::string &content, const std::string &rheader) {
				fmt post =
					"POST {1} HTTP/1.0\n"
					"HOST: {0}\n"
					"Content-type: application/json\n"
					"Content-length: {2}\n{3}\n{4}\n";

				post.arg(base).arg(url).arg(content.length()).arg(rheader).arg(content);

				run(fmt(R"(AT+CIPSTART="TCP","{0}",80)").arg(base)).then([this, &post](auto) {
					run("AT+CIPSEND=" + std::to_string(post.length()));
				});
			}
		};
	\end{lstlisting}
\end{latin}

\subsection{کلاس \lr{utils}}
کلاس
\lr{utils}
برخی از ابزارهای مورد نیاز برای راه‌اندازی \lr{STM32}، مانند تنظیم \lr{PLL} و \lr{RTC} و همچنین تابع \lr{delay}.
در این بین تابع \lr{delay} کمی متفاوت‌تر از توابع موجود پیاده‌سازی شده، بدین صورت که در صورت وجود مقدار مکث بیشتر از
$10$
ثانیه، پردازنده به خواب رفته و یک \lr{timer} برای خارج کردن پردازنده از خواب شروع می‌شود.
بدین جهت که در مکث‌های بیشتر از یک یا دو دقیقه مصرف انرژی کمتری توسط پردازنده مصرف شود.

آدرس فایل:
\hyperref{https://github.com/cardianco/cardian-hardware/blob/master/main/mcu/mcu.h}{}{}{\lrm{cardian-hardware/hardware/main/mcu/mcu.h}}
\begin{latin}
	\small
	\begin{lstlisting}[language=c++, caption={MCU tool functions}]
		/// Implementation details of the main code.
		struct utils final {
			inline static uint64_t utc = 0;

			static void init() { NVIC_SetPriorityGrouping(4); }

			static inline void setupHSI() {
				RCC->CR |= RCC_CR_HSION;
				while(!(RCC->CR & RCC_CR_HSIRDY));
				RCC->CFGR &= ~RCC_CFGR_SW;
				while((RCC->CFGR & RCC_CFGR_SWS) != RCC_CFGR_SWS_HSI);
			}

			static inline void pllHsiTo32mhz() {
				RCC->CR &= ~RCC_CR_PLLON;
				while(RCC->CR & RCC_CR_PLLRDY);
				RCC->CFGR &= ~RCC_CFGR_PLLMULL;
				RCC->CFGR |= RCC_CFGR_PLLMULL8;

				RCC->CR |= RCC_CR_PLLON;
				while(!(RCC->CR & RCC_CR_PLLRDY));
				RCC->CFGR &= ~RCC_CFGR_SW;
				RCC->CFGR |= RCC_CFGR_SW_PLL;
				while((RCC->CFGR & RCC_CFGR_SWS) != RCC_CFGR_SWS_PLL);

				SystemCoreClockUpdate();
			}

			static void delay(uint32_t ms = UINT32_MAX) {
				if(ms < 5000) {
					__disable_irq();

					RCC->APB1ENR |= RCC_APB1ENR_TIM4EN;
					TIM4->PSC = SystemCoreClock/1000;
					TIM4->EGR |= TIM_EGR_UG;
					TIM4->CNT = 0;
					TIM4->ARR = UINT32_MAX;
					TIM4->CR1 |= TIM_CR1_CEN;

					while(TIM4->CNT < ms) __asm("nop");

					__enable_irq();
				} else {
					TIM4->CR1 &= ~(TIM_CR1_CEN);
					RCC->APB1ENR |= RCC_APB1ENR_TIM4EN;

					NVIC_SetPriority(TIM4_IRQn, 0);
					NVIC_EnableIRQ(TIM4_IRQn);

					TIM4->PSC = SystemCoreClock/1000;
					TIM4->EGR |= TIM_EGR_UG;
					TIM4->ARR = ms;

					TIM4->DIER |= TIM_DIER_UIE;
					TIM4->CR1 |= TIM_CR1_CEN;

					__WFI();

					NVIC_DisableIRQ(TIM4_IRQn);
					RCC->APB1ENR &= ~RCC_APB1ENR_TIM4EN;
				}
			}

			static uint32_t timxNum(TIM_TypeDef *timx) {
				return (timx == TIM1) * 0x1 | (timx == TIM2) * 0x2 |
				(timx == TIM3) * 0x3 | (timx == TIM4) * 0x4;
			}

			template <uint32_t num = 1, bool interrupt = false>
			static void timxInit() {
				static_assert( 1 <= num && num <= 4, "Timer number should be between 1 and 4");
				timxInit(num - 1, interrupt);
			}

			static void timxInit(uint32_t idx, bool interrupt = false) {
				IRQn_Type timx_irqn[4] = {TIM1_UP_IRQn, TIM2_IRQn, TIM3_IRQn, TIM4_IRQn};
				TIM_TypeDef* timx[4] = {TIM1, TIM2, TIM3, TIM4};
				uint32_t timx_en[4] = {RCC_APB2ENR_TIM1EN, RCC_APB1ENR_TIM2EN, RCC_APB1ENR_TIM3EN, RCC_APB1ENR_TIM4EN};

				timx[idx]->CR1 &= ~(TIM_CR1_CEN);
				if(idx) RCC->APB1ENR |= timx_en[idx];
				else RCC->APB2ENR |= timx_en[idx];

				NVIC_SetPriority(timx_irqn[idx], idx);
				NVIC_EnableIRQ(timx_irqn[idx]);

				timx[idx]->PSC = SystemCoreClock/1000;
				timx[idx]->EGR |= TIM_EGR_UG;
				timx[idx]->ARR = UINT32_MAX;
				timx[idx]->DIER |= interrupt ? TIM_DIER_UIE : 0x0UL;
			}

			void initRtc(void) {
				RCC->APB1ENR |= RCC_APB1ENR_PWREN;
				PWR->CR |= PWR_CR_DBP;
				RCC->BDCR |= RCC_BDCR_RTCEN;
				RCC->BDCR |= RCC_BDCR_RTCSEL_LSE;
				RCC->BDCR &= ~RCC_BDCR_RTCSEL_LSI;
				RCC->BDCR |= RCC_BDCR_LSEON;
				while(!(RCC->BDCR & RCC_BDCR_LSERDY));
				RTC->CRL |= RTC_CRL_CNF;
				RTC->CRH |= RTC_CRH_SECIE;
				RTC->CRL &= ~RTC_CRL_CNF;
				NVIC_EnableIRQ(RTC_IRQn);
			}
		};
	\end{lstlisting}
\end{latin}
